% Online-Material zu Teil 1 und Teil 2 meiner Tipps, Tricks und
% Techniken Videoserie auf YouTube
%
\documentclass[11pt,a4paper]{scrartcl}
\usepackage[T1]{fontenc}
\usepackage[utf8]{inputenc}
\usepackage[ngerman]{babel}
\usepackage{tabularx}
\usepackage{booktabs}

% Die folgenden Pakete sind zum Setzen der Tastatursymbole
% notwendig:
\usepackage{graphicx}
\usepackage{tikz}
\usetikzlibrary{backgrounds}
\usetikzlibrary{calc}

% Für Links zu den YouTube-Videos.
\usepackage{hyperref}

% Kommando keystroke setzt ein Tastatursymbol mit dem Argument als Inhalt:
\newcommand*\keystroke[1]{%
  \begin{tikzpicture}[baseline=(key.base), very thin, line cap=round, black, rounded corners=0pt]%
    \node [draw, fill=white, fill opacity=1, rectangle, rounded corners=2pt, inner sep=1pt, minimum width=1.2em, font=\scriptsize\sffamily] (key) {#1\strut};

    \begin{scope}[on background layer]
      \draw [rounded corners=1pt, fill=white] ($ (key.north west) + (-2pt, 2pt) $) rectangle ($ (key.south east) + (2pt, -2pt) $);

      \fill [gray!60] ($ (key.south west) + (2pt, 0.1pt) $) -- ($ (key.south west) + (-1pt, -2pt) $)
                  -- ($ (key.south east) + (1pt, -2pt) $)  -- ($ (key.south east) + (-2pt, 0.1pt) $) -- cycle;

      \fill [gray!60] ($ (key.south east) + (-0.1pt, 2pt) $) -- ($ (key.south east) + (2pt, -1pt) $)
                  -- ($ (key.north east) + (2pt, 1pt) $)    -- ($ (key.north east) + (-0.1pt, -2pt) $) -- cycle;
    \end{scope}

    \draw ($ (key.north west) + (0.1pt, -2pt) $) -- ($ (key.north west) + (-2pt, 1pt) $);
    \draw ($ (key.north west) + (2pt, -0.1pt) $) -- ($ (key.north west) + (-1pt, 2pt) $);

    \draw ($ (key.north east) + (-0.1pt, -2pt) $) -- ($ (key.north east) + (2pt, 1pt) $);
    \draw ($ (key.north east) + (-2pt, -0.1pt) $) -- ($ (key.north east) + (1pt, 2pt) $);

    \draw ($ (key.south west) + (0.1pt, 2pt) $) -- ($ (key.south west) + (-2pt, -1pt) $);
    \draw ($ (key.south west) + (2pt, 0.1pt) $) -- ($ (key.south west) + (-1pt, -2pt) $);

    \draw ($ (key.south east) + (-0.1pt, 2pt) $) -- ($ (key.south east) + (2pt, -1pt) $);
    \draw ($ (key.south east) + (-2pt, 0.1pt) $) -- ($ (key.south east) + (1pt, -2pt) $);
  \end{tikzpicture}%
}

% Windows Logo zum Setzen der Windows-Taste:
\newcommand{\WindowsLogo}{\raisebox{-0.1em}{%
  \includegraphics[height=0.8em]{Windows_3_logo_simplified}}}

\begin{document}
%
\section*{Windows Tastaturkürzel zur effektiven Fensterhandhabung
-  unter Windows 10}
Wir fassen unten die Techniken zur effektiven Verwaltung von Anwendungen aus \href{https://youtu.be/wjzZGac1yso}{Teil~01} und Teil~02 meiner \href{https://youtube.com/playlist?list=PL0FqMC_xCtjQCsFwN8ci8bJhci5xMx1Bm}{\emph{Tipps, Tricks und Techniken zu Windows 10}-Videoserie} auf \texttt{YouTube} zusammen.
%
\subsection*{Teil~01: Anwendungen auswählen und Windows-Snapping}
\label{subsec:windows_snapping}
%
Mit den folgenden Tastaturkürzeln können Sie einfach und effektiv
aktive Windows-Anwendungen auswählen oder am Bildschirm anordnen
(Windows-Snapping). Die entsprechenden Schritte wurden in
\href{https://youtu.be/wjzZGac1yso}{Teil~01} der Videoserie behandelt.
%
\begin{center}
  \begin{tabularx}{0.9\textwidth}{lX}
  \toprule
  Tastaturkürzel & Kurze Erläuterung \\
  \midrule
  % 
  \keystroke{Alt}+\keystroke{Tab} & Erlaubt das Wechseln zwischen offenen   Anwendungen (wiederholtes Drücken von \texttt{Tab}, während \texttt{Alt} gedrückt bleibt).\\
  %
  \keystroke{Ctrl}+\keystroke{Alt}+\keystroke{Tab} & Ähnlich zu
  \keystroke{Alt}+\keystroke{Tab}, aber die Thumbnails bleiben erhalten,
  nachdem man alle Tasten loslässt; wurde nicht explizit im Video besprochen. \\
  \midrule 
  %
  \keystroke{\WindowsLogo}+\keystroke{$\leftarrow$} & Hefte die Anwendung im Vordergrund an den linken Teil des Bildschirms an. \\
  %
  \keystroke{\WindowsLogo}+\keystroke{$\rightarrow$} & Hefte die Anwendung im Vordergrund an den rechten Teil des Bildschirms an. \\
  \midrule
  %
  \keystroke{\WindowsLogo}+\keystroke{$\uparrow$} & Der Effekt ist
  kontextabhängig: 1. Hefte die Anwendung im Vordergrund an den oberen
  (linken/rechten) Teil des Bildschirms an (nach
  \keystroke{\WindowsLogo}+\keystroke{$\leftarrow$} oder
  \keystroke{\WindowsLogo}+\keystroke{$\rightarrow$}); 2. Vergrößere
  eine unten angeheftete Anwendung auf den gesamten (linken/rechten)
  Teil des Bildschirms; 3. Ausserhalb des Windows-Snapping führt die
  Tastenkombination zur Maximierung der Fenstergröße (wurde im Video
  nicht explizit gezeigt). \\
  %
  \keystroke{\WindowsLogo}+\keystroke{$\downarrow$} & Der Effekt ist
  kontextabhängig: 1. Hefte die Anwendung im Vordergrund an den unteren
  (linken/rechten) Teil des Bildschirms an (nach
  \keystroke{\WindowsLogo}+\keystroke{$\leftarrow$} oder
  \keystroke{\WindowsLogo}+\keystroke{$\rightarrow$}); 2. Vergrößere
  eine oben angeheftete Anwendung auf den gesamten (linken/rechten) Teil
  des Bildschirms; 3. Ausserhalb des Windows-Snapping führt die
  Tastenkombination zur Minimierung der Anwendung (wurde im Video nicht
  explizit gezeigt). \\
  \bottomrule
  \end{tabularx}
\end{center}
%
\subsection*{Teil~02: Virtuelle Desktops}
\label{subsec:virtuelle_desktops}
%
Virtuelle Desktops werden in Teil~02 der Videoserie vorgestellt. Hier alle wesentlichen Tastaturkürzel zu dem Thema: 
%
\begin{center}
  \begin{tabularx}{0.9\textwidth}{lX}
  \toprule
  Tastaturkürzel & Kurze Erläuterung \\
  \midrule
  % 
  \keystroke{\WindowsLogo}+\keystroke{Tab} & Wechseln zur Taskansicht des aktiven Desktops. Hier können unter anderem neue virtuelle Desktops mit der Maus erzeugt oder alte gelöscht werden. Anwendungen des aktiven Desktops lassen sich per Drag \& Drop in andere Desktops verschieben. \\
  %
  \keystroke{Esc} & Taskansicht verlassen \\	
  \midrule 
  %
  \keystroke{Ctrl}+\keystroke{\WindowsLogo}+\keystroke{$\leftarrow$} &
  zum linken virtuellen Desktop wechseln. \\
  %
  \keystroke{Ctrl}+\keystroke{\WindowsLogo}+\keystroke{$\rightarrow$} & zum rechten virtuellen Desktop wechseln \\
  \midrule
  %
  \keystroke{Ctrl}+\keystroke{\WindowsLogo}+\keystroke{d} & neuen virtuellen Desktop erzeugen und in diesen wechseln \\
  %
  \keystroke{Ctrl}+\keystroke{\WindowsLogo}+\keystroke{F4} & aktuellen Desktop löschen. Alle offenen Anwendungen werden in einen benachbarten Desktop verschoben und in diesen wird gewechselt. \\
  %
  \bottomrule
  \end{tabularx}
\end{center}
\end{document}
