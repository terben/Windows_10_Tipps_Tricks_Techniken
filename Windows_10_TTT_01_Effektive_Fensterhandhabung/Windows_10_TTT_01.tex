\documentclass[11pt,a4paper]{scrartcl}
\usepackage[T1]{fontenc}
\usepackage[utf8]{inputenc}
\usepackage{tabularx}
\usepackage{booktabs}

% Die folgenden Pakete sind zum Setzen der Tastatursymbole
% notwendig:
\usepackage{graphicx}
\usepackage{tikz}
\usetikzlibrary{backgrounds}
\usetikzlibrary{calc}

% Für Links zu den YouTube-Videos.
\usepackage{hyperref}

% Kommando keystroke setzt ein Tastatursymbol mit dem Argument als Inhalt:
\newcommand*\keystroke[1]{%
  \begin{tikzpicture}[baseline=(key.base), very thin, line cap=round, black, rounded corners=0pt]%
    \node [draw, fill=white, fill opacity=1, rectangle, rounded corners=2pt, inner sep=1pt, minimum width=1.2em, font=\scriptsize\sffamily] (key) {#1\strut};

    \begin{scope}[on background layer]
      \draw [rounded corners=1pt, fill=white] ($ (key.north west) + (-2pt, 2pt) $) rectangle ($ (key.south east) + (2pt, -2pt) $);

      \fill [gray!60] ($ (key.south west) + (2pt, 0.1pt) $) -- ($ (key.south west) + (-1pt, -2pt) $)
                  -- ($ (key.south east) + (1pt, -2pt) $)  -- ($ (key.south east) + (-2pt, 0.1pt) $) -- cycle;

      \fill [gray!60] ($ (key.south east) + (-0.1pt, 2pt) $) -- ($ (key.south east) + (2pt, -1pt) $)
                  -- ($ (key.north east) + (2pt, 1pt) $)    -- ($ (key.north east) + (-0.1pt, -2pt) $) -- cycle;
    \end{scope}

    \draw ($ (key.north west) + (0.1pt, -2pt) $) -- ($ (key.north west) + (-2pt, 1pt) $);
    \draw ($ (key.north west) + (2pt, -0.1pt) $) -- ($ (key.north west) + (-1pt, 2pt) $);

    \draw ($ (key.north east) + (-0.1pt, -2pt) $) -- ($ (key.north east) + (2pt, 1pt) $);
    \draw ($ (key.north east) + (-2pt, -0.1pt) $) -- ($ (key.north east) + (1pt, 2pt) $);

    \draw ($ (key.south west) + (0.1pt, 2pt) $) -- ($ (key.south west) + (-2pt, -1pt) $);
    \draw ($ (key.south west) + (2pt, 0.1pt) $) -- ($ (key.south west) + (-1pt, -2pt) $);

    \draw ($ (key.south east) + (-0.1pt, 2pt) $) -- ($ (key.south east) + (2pt, -1pt) $);
    \draw ($ (key.south east) + (-2pt, 0.1pt) $) -- ($ (key.south east) + (1pt, -2pt) $);
  \end{tikzpicture}%
}

% Windows Logo zum Setzen der Windows-Taste:
\newcommand{\WindowsLogo}{\raisebox{-0.1em}{%
  \includegraphics[height=0.8em]{Windows_3_logo_simplified}}}

\begin{document}
%
\section*{Windows Tastaturkürzel zur effektiven Fensterhandhabung}
%
\subsection*{Anwendungen auswählen und Windows-Snapping}
Mit den folgenden Tastaturkürzeln können Sie einfach und effektiv aktive
Windows-Anwendungen auswählen und am Bildschirm anordnen (Windows-Snapping). Ich zeige die entsprechenden Schritte auch in Teil~01 meiner \emph{Tipps, Tricks und Techniken zu Windows 10}-Videoserie auf \texttt{YouTube}. 
%
\begin{center}
\begin{tabularx}{0.9\textwidth}{lX}
\toprule
Tastaturkürzel & Kurze Erläuterung \\
\midrule
%
\keystroke{Alt}+\keystroke{Tab} & Erlaubt das Wechseln zwischen offenen   Anwendungen (wiederholtes Drücken von \texttt{Tab}, während \texttt{Alt} gedrückt bleibt).\\
%
\keystroke{Ctrl}+\keystroke{Alt}+\keystroke{Tab} & Ähnlich zu   \keystroke{Alt}+\keystroke{Tab}, aber die Thumbnails bleiben erhalten, nachdem man alle Tasten loslässt. \\
\midrule
%
\keystroke{\WindowsLogo}+\keystroke{$\leftarrow$} & Hefte die Anwendung im Vordergrund an den linken Teil des Bildschirms an. \\
%
\keystroke{\WindowsLogo}+\keystroke{$\rightarrow$} & Hefte die Anwendung im Vordergrund an den rechten Teil des Bildschirms an. \\
\midrule
%
\keystroke{\WindowsLogo}+\keystroke{$\uparrow$} & Hefte die Anwendung im Vordergrund an den oberen (linken/rechten) Teil des Bildschirms an (nach \keystroke{\WindowsLogo}+\keystroke{$\leftarrow$} oder \keystroke{\WindowsLogo}+\keystroke{$\rightarrow$}). \\
%
\keystroke{\WindowsLogo}+\keystroke{$\downarrow$} & Hefte die Anwendung im Vordergrund an den unteren (linken/rechten) Teil des Bildschirms an (nach \keystroke{\WindowsLogo}+\keystroke{$\leftarrow$} oder \keystroke{\WindowsLogo}+\keystroke{$\rightarrow$}). \\
\bottomrule
\end{tabularx}
\end{center}	
\end{document}

